\documentclass[12pt]{article}

\usepackage[margin=1.25in]{geometry}

\author{\\ Burenko Anton \\ s76905 \\ \\ \\ Prof. Dr. Wolfgang Oertel \\ 
Computergrafik 1 \\ \\
}

\date{Wintersemester 2018/19}
\title{Belegarbeit: "Sunset Pagoda"}

\begin{document}

\maketitle

\pagebreak

\tableofcontents

\pagebreak

\section{Aufgabenstellung}
Schreiben Sie ein Programm in C/C++, das unter Verwendung von OpenGL, Vertex- und
Fragment-Shadern folgende Aufgaben realisiert. \\ \\
\textbf{Aufgabe 1}: \\
Geometrische Objekte: Erzeugen Sie eine interaktive zeitlich animierte Szene mit mehreren
unterschiedlichen farblichen und texturierten dreidimensionalen geometrischen Objekten. \\ \\
\textbf{Aufgabe 2}: \\
Beleuchtung: Beleuchten Sie die Szene mit mehreren Lichtquellen so, dass auf den Objekten
unterschiedliche Beleuchtungseffekte sichtbar werden. \\ \\
\textbf{Aufgabe 3}: \\
Ansicht: Stellen Sie die Szene gleichzeitig in verschiedenen Ansichten und Projektionen in
mehreren Viewports des Anzeigefensters dar. \\ \\
\textbf{Aufgabe 4}: \\
Programm: Stellen Sie das komplette Programm in Quelltextform als Visual-Studio-C++ -
Projekt und in ausführbarer Form als exe-File derart bereit, dass die Lauffähigkeit auf den
Computern des Praktikumslabors der Lehrveranstaltung gewährleistet ist. \\ \\
\textbf{Aufgabe 5}: \\
Dokumentation: Fertigen Sie eine Systemdokumentation in Form eines pdf-Dokumentes von
etwa 10 Seiten an, die Deckblatt, Gliederung, Aufgabenbeschreibung, Lösungsansatz,
Lösungsumsetzung, Installations- und Bedienungsanleitung, einige Bildschirm-Snapshots,
Probleme, Ergebnisse, Literatur- und Quellenverzeichnis enthält. \\ \\
\textbf{Aufgabe 6}: \\
Abgabe: Demonstrieren Sie die Ergebnisse der Aufgaben 4 und 5 an einem Computer des
Praktikumslabors der Lehrveranstaltung und übergeben Sie diese in einem Verzeichnis
$"Name\_Vorname\_Bibliotheksnummer"$. \\ \\
\textbf{Zeitplan}: \\
Die Ausgabe der Aufgabenstellung erfolgt zu Beginn der Lehrveranstaltungszeit. Die Abgabe
der Ergebnisse erfolgt spätestens zum Ende der Lehrveranstaltungszeit. \\

\pagebreak

\section{NEW CHAPTER}



\end{document}
\pagebreak
 